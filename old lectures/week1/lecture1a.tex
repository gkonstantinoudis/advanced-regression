\documentclass{beamer}

\usetheme{Montpellier}
%\usetheme{CambridgeUS}


\usepackage[OT1]{fontenc}
\usepackage[utf8x]{inputenc}


%\usepackage[T1]{fontenc}
%\usepackage[latin9]{inputenc}
\usepackage[german]{babel}
\usepackage{booktabs,bm, color, enumerate, hyperref, pgf, url, soul, tikz}
\usepackage{amssymb, amsmath}
\usepackage{graphicx}
\newcommand*{\Scale}[2][4]{\scalebox{#1}{$#2$}}%
\newcommand*{\Resize}[2]{\resizebox{#1}{!}{$#2$}}%

%\usepackage{hyperref}

\usepackage{appendixnumberbeamer}


\bibliographystyle{apalike}



%\input{abbreviations}



\setbeamertemplate{blocks}[rounded][shadow=true]
%\usepackage{appendixnumberbeamer}



%\subject{2015}




\title{Advanced Regression: 1a Overview of the course}



\author{Garyfallos Konstantinoudis}
\institute{Epidemiology and Biostatistics, Imperial College London}




\date{21 February 2023}


\setlength{\unitlength}{0.9cm}
\linethickness{1.1pt}








\begin{document}


\frame{
\titlepage
}



\frame{
\tableofcontents
}

%\section{Advanced Regression: Motivation}
%\section{Advanced Regression: Course aims and motivation}
%\section{Advanced Regression: Practicals}
%\section{Advanced Regression: Timetable}
%\section{Advanced Regression: Questions?}




\section{Advanced Regression: Motivation}
\subsection{Course aims}

\frame{
\frametitle{Advanced Regression: Course aims} 


\centering \includegraphics [scale=0.3]{E:/Postdoc Imperial/Lectures/2023_AdvancedRegression/AdvancedRegression2022-2023/lectures/graphs/ML-omics}



\begin{block}{}{
\begin{itemize}
\item Learn principles of advanced regression for high-dimensional data analysis.
\item Apply these techniques on real-world data problems.
\end{itemize}}
\end{block}

}


\subsection{High-dimensional data}

\frame{
\frametitle{Motivation: High-dimensional data} 

\begin{block}{}{
\begin{itemize}
\item Number of samples or observations: $n$
\item Number of variables: $p$
\end{itemize}
}
\end{block}


Data types:
\begin{itemize}
\item Big data: $p >> n$ \\
\end{itemize}
\includegraphics [scale=0.25]{E:/Postdoc Imperial/Lectures/2023_AdvancedRegression/AdvancedRegression2022-2023/lectures/graphs/bigdata} 
    \begin{columns}[onlytextwidth,T]
      \column{\dimexpr\linewidth-30mm-5mm}
\begin{itemize}
\item(Tall data: Summary-level data $p \times 1$)
\end{itemize}
      \column{30mm}
	\vspace{-1.4cm}
      \includegraphics[scale=0.25]{E:/Postdoc Imperial/Lectures/2023_AdvancedRegression/AdvancedRegression2022-2023/lectures/graphs/talldata}
    \end{columns}


}




\frame{
\frametitle{Examples high-dimensional data types} 

\begin{itemize}
\item Health data records, e-records
\item Health and fitness apps, location tracker
\item Imaging data, e.g. functional and structural fMRI studies
\item Credit scoring based on credit files and personal data
\item Recommender systems based on user ratings
\end{itemize}

\begin{block}{ Modern data science is build on advanced regression models!}{

}
\end{block}
}


\frame{
\frametitle{Methods covered in this course} 

\begin{itemize}
\item Random effects and hierarchical models
\item Non-linear regression
\item Penalised regression (Ridge, lasso, and elastic net)
\item Building a prediction rule and cross-validation
\item Classification with discriminant analysis and support vector machines
\item Non-parametric methods (bagging, boosting, decision trees and random forests)
\item Machine-learning models (Neural networks)
\end{itemize}

}




\section{Advanced Regression: Course details}

\subsection{Practicals}

\frame{
\frametitle{Advanced Regression: Practicals} 


\begin{itemize}
\item Structure: First lectures, then practicals and homework (optional) with one week delay

\vspace{-.1cm}

\begin{center}
\includegraphics [scale=0.27]{E:/Postdoc Imperial/Lectures/2023_AdvancedRegression/AdvancedRegression2022-2023/lectures/graphs/Structure2023.png}
\end{center}

\vspace{-1.1cm}

\item Practical questions are available on blackboard. Solutions available online after the practical.  
\item Open discussion with module lead and tutors
\end{itemize} 


}

\frame{
\frametitle{Advanced Regression: Practicals} 


\begin{block}{Each student needs to present the solution to one practical question.}{
 \begin{itemize}
\item It is required to present one practical question to be admitted  to the exam. ({\color{red} not this year})
\end{itemize}
}
\end{block}

\begin{block}{Practicals are an essential part of the course.}{
 \begin{itemize}
\item They help you to understand better the content and the relevance of the topics presented in the lectures. 
\item They are an important preparation for the exam.
\item But more importantly, you will learn the basics of data science and statistical computing.
\end{itemize}
}
\end{block}


}


\frame{
\frametitle{Why use R?} 



\centering \includegraphics [scale=0.06]{E:/Postdoc Imperial/Lectures/2023_AdvancedRegression/AdvancedRegression2022-2023/lectures/graphs/Rlogo}


\begin{itemize}
\item The practicals will be in R.
\item R is a language and environment for statistical computing and graphics.  \\
\item R is free and published under the GNU licence.
\item 16,883 available add-on packages.
\item Please download R and be prepared to run analysis before the practicals.
\item Practical questions will be posted on Blackboard Thursday or Friday before the practical.
\item  \url{https://cran.r-project.org/}
\end{itemize}

}


\frame{
\frametitle{What is markdown?} 

\begin{itemize}
\item  When coding in R it is important to document and comment the code.
\item  Markdown is an R package that compiles R code into documents (pdf, html, word and many more). 
\item Package
\url{https://cran.r-project.org/web/packages/rmarkdown}
\item Project page
\url{https://rmarkdown.rstudio.com/}
\end{itemize}

\begin{block}{Make your code accessible and reproducible.}{
\begin{itemize}
\item Markdown can help you with that.
\item Both practical questions and solutions will be provided in markdown and pdf format.
\end{itemize}
}
\end{block}

}





\subsection{Timetable}

\frame{
\frametitle{Week 1: 21st February} 
\begin{tabular}{l l l l}
  \hline	
  \hline			
10:00-10:20 & Lecture 1a & Overview  & GK \\
	& 		& and motivation  & \\
  \hline
10:20-11:00 & Lecture 1b & Linear and generalised  & GK \\
& 		& linear models  & \\
\hline
11:10-12:00 & Lecture 1c &  Random effects and  & GK \\
  &     & hierarchical models  & \\
  \hline
  \hline
  13:00-15:00 & Practical 1 & Using R to analyse data  &  GK \& Christina \\
  & 		& with linear models  & \\
  \hline 
  \hline
\end{tabular}
}

\frame{
\frametitle{Week 2: 28th March} 
\begin{tabular}{l l l l}
  \hline	
  \hline	
10:00-10:50 & Lecture 2a & Introduction to non-linear  & GK \\
	& 		& regression  & \\
  \hline
11:00-11:50 & Lecture 2b & Bias and variance trade off   & GK \\
	& 		& and  penalised splines   & \\
  \hline
12:00-12:50 & Lecture 2c & Distributed non-linear & GK \\
	& 		&  lag models  & \\
  \hline
  \hline
  14:00-16:00 & Practical 2 & Using R to perform non-linear  &  GK \& CL \\
& 		& regression  & \\
  \hline
  \hline 
\end{tabular}
}



\frame{
\frametitle{Week 3: 7th March} 
\begin{tabular}{l l l l}
	\hline	
	\hline	
	10:00-10:50 & Lecture 2a & Variable selection  & GK \\
	\hline
	11:00-11:50 & Lecture 2b & Prediction accuracy and   & GK \\
		& 		& cross-validation  & \\
	\hline
	12:00-12:50 & Lecture 2c & Penalised regression models & GK \\
	\hline
	\hline
	14:00-16:00 & Practical 3 & Using R to perform  &  GK \& CL \\
		& 		&  cross-validation and  & \\
	& 		& penalised regression  & \\
	\hline
	\hline 
\end{tabular}
}



\frame{
\frametitle{Week4:  14st March} 
\begin{tabular}{l l l l}
	\hline	
\hline	
10:00-10:50 & Lecture 2a & Machine learning:  & GK \\
& 		&  Classification  & \\
\hline
11:00-11:50 & Lecture 2b & Machine learning:    & GK \\
& 		&  Ensemble methods & \\
\hline
12:00-12:50 & Lecture 2c & Machine learning: & GK \\
& 		&  Neural networks & \\
\hline
\hline
14:00-16:00 & Practical 4 & Using R to perform  &  GK \& CL \\
& 		&  classification methods  & \\
\hline
\hline 
\end{tabular}
}




\frame{
\frametitle{Week 5: 21th March} 
\begin{tabular}{l l l l}
	\hline	
\hline	
10:00-12:00 & Practical 5 & Using R to understand  &  GK \& CL \\
& 		&  ensemble methods  & \\
\hline
\hline
13:00-14:00 & Mock exam & Go through the mock   & GK \\
& 		& exam  & \\
\hline
\hline
14:00-15:00 & Q\&A Session & Revisit concepts/lectures & GK \\
\hline
\hline
\end{tabular}
}




\subsection{Learning outcomes and exam}

\frame{
\frametitle{Advanced Regression: Learning outcomes} 
\begin{itemize}
\item	Perform advanced statistical analyses, employing penalised likelihood or non-pararametric regression models.
\item	Discuss the theoretical foundations and limitations of the most widely used advanced regression approaches.
\item	Identify the challenges of high-dimensional data analysis.
\item	Identify suitable analysis strategies to address the problems arising from 'small $n$, large $p$' data sets.
\item	Use complex regression models in R, understand which methods are suitable for which data, 
know the pitfalls of high-dimensional data analysis, and interpret the results.
\item Enjoy data science.
\end{itemize}

}


\frame{
\frametitle{Advanced Regression: Exam} 

\begin{itemize}
\item This module will be assessed by a written open book programming exam which is a mixture of coding in \texttt{R}, interpretion of outputs, and description of why and how the analysis is performed.
\item The exam is taking place in \textcolor{red}{early May 2023 (tbc)}. 
\item Practical sessions offer regular opportunity for receiving formative feedback from the tutors. 
\item A mock exam will be provided and discussed. 
\item A Q $\&$ A session will be scheduled on the last day of the module and if helpful before the exam (End of April).
\end{itemize}

}



\subsection{Questions?}

\frame{
\frametitle{Advanced Regression: Questions?} 


	
\begin{block}{Please get in touch:}{
\href{mailto:g.konstantinoudis@imperial.ac.uk}{g.konstantinoudis@imperial.ac.uk} 
}
\end{block}

\vspace{0.8cm}

\begin{block}{Blackboard:}{
		Discussion board, you can also post anonymously. 
	}
\end{block}

\vspace{0.8cm}
\begin{block}{Drop-in sessions:}{
Every Wednesday 16:00-17:00 UK time
}
\end{block}

Zoom link will also be provided.


}




\frame{
\frametitle{Next lectures} 

\textbf{LECTURE 1b Linear models and generalised linear models}
\begin{itemize}
\item Repetition: The linear model
\item Generalised linear model
\end{itemize}
\textbf{LECTURE 1c Random effects models}
\begin{itemize}
\item Motivation: Structured data
\item Fixed and random effects
\end{itemize}


}


\end{document}




