\documentclass{beamer}

\usetheme{Montpellier}
%\usetheme{CambridgeUS}


\usepackage[OT1]{fontenc}
\usepackage[utf8x]{inputenc}


%\usepackage[T1]{fontenc}
%\usepackage[latin9]{inputenc}
\usepackage[german]{babel}
\usepackage{booktabs,bm, color, enumerate, hyperref, pgf, url, soul, tikz}
\usepackage{amssymb, amsmath}
\usepackage{graphicx}
\newcommand*{\Scale}[2][4]{\scalebox{#1}{$#2$}}%
\newcommand*{\Resize}[2]{\resizebox{#1}{!}{$#2$}}%

%\usepackage{hyperref}

\usepackage{appendixnumberbeamer}


\bibliographystyle{apalike}



%\input{abbreviations}



\setbeamertemplate{blocks}[rounded][shadow=true]
%\usepackage{appendixnumberbeamer}



%\subject{2015}




\title{Advanced Regression: 2a Variable selection}



\author{Garyfallos Konstantinoudis}
\institute{Epidemiology and Biostatistics, Imperial College London}




\date{7th March 2023}


\setlength{\unitlength}{0.9cm}
\linethickness{1.1pt}








\begin{document}


\frame{
\titlepage
}



\frame{
\tableofcontents
}

%\section{Variable ranking, variable importance and variable selection}
%\section{Classical variable or model selection}
%\section{Variable importance}
%\section{Variable ranking}
%\section{Example: Genome-wide association studies}
%\section{Example: Differential expression}



\section{Classical variable or model selection}

\frame{
\frametitle{Variable or model selection} 


\begin{block}{Variable or model selection}
To select a model (a set of variables, i.e. one or many variables) jointly.
\end{block}


\begin{itemize}
\item Focus is not on a single variable but on a model, i.e. one or a combination of many variables.
\item Motivation: \textbf{To understand} which combination of variables explains best the outcome and \textbf{to predict} future outcomes.
\end{itemize}




}





\frame{
\frametitle{Classical variable or model selection} 

Measures used to compare models:
\begin{itemize}
\item Proportion of variance explained
\item $F-$statistic and analysis of variances (ANOVA)
\item Likelihood based methods
\begin{itemize}
\item Likelihood ratio test (LRT)	
\item Akaike information criterion (AIC)
\item Bayesian information criterion (BIC)
\end{itemize}
\end{itemize}

}


\subsection{Variance explained and ANOVA}


\frame{
\frametitle{Proportion of variance explained} 


\begin{block}{Linear model}{
\begin{equation}
y =  x \beta + \epsilon \nonumber
\end{equation}}
\end{block}


\begin{itemize}
\item Variance decomposition:
\begin{equation}
\underbrace{var(y)}_{\text{Total Variance}} = \underbrace{var(x\beta)}_{\text{Explained Variance}} + \underbrace{var(\epsilon)}_{\text{Error Variance}} \nonumber
\end{equation}
\item $R^2$ is the proportion of variance explained by a model
\begin{equation}
R^2 = \frac{var(x\beta)}{var(y)} = 1-\frac{var(\epsilon)}{Var(y)} \nonumber
\end{equation}
\end{itemize}




}



\frame{
\frametitle{Proportion of variance explained} 

\begin{itemize}
\item How to compute? Using sum of squares (SS):
\begin{itemize}
\item[$\diamond$] Total variance
\begin{equation}
\hat{var}(y) = \frac{1}{n-1} SS_{Total} = \frac{1}{n} \sum_{i=1}^n[y_i - \bar{y}]^2  \nonumber
\end{equation}
\item[$\diamond$] Explained variance $\hat{y}=x\beta$
\begin{equation}
\hat{var}(\hat{y}) = \frac{1}{n-1} SS_{Explained} = \frac{1}{n} \sum_{i=1}^n[\hat{y}_i - \bar{y}]^2  \nonumber
\end{equation}
\item[$\diamond$] Error variance
\begin{equation}
\hat{var}(\epsilon) = \frac{1}{n-1} SS_{Error} = \frac{1}{n} \sum_{i=1}^n[y_i -\hat{y}_i]^2  \nonumber
\end{equation}
where the mean is defined as $\bar{y} = \frac{1}{n} \sum_{i=1}^n y_i  $
\end{itemize}
\end{itemize}

}

\frame{
\frametitle{Occam's razor} 

\begin{itemize}
\item When comparing two models it is important not only to consider $R^2$ but also how complex they are, i.e. how many variables they include.
\item Occam's razor (law of parsimony): Simpler solutions are more likely to be correct than complex ones (William of Ockham 1287–1347ad).
\item Problem: $R^2$ will always increase when including more variables.
\item Question: Is including more variables actually improving the model fit significantly?
\end{itemize}



}


\frame{
\frametitle{Adjusted proportion of variance explained} 

\begin{itemize}
\item Adjusted $R^2$
\begin{equation}
R^2_{adj} = 1- (1-R) \times \frac{n-1}{n-p-1} \nonumber
\end{equation}
\item Alternative representation with degrees of freedom (df)
\begin{equation}
R^2_{adj} = 1- \frac{SS_{Error}/df_e}{SS_{Total}/df_t} \nonumber
\end{equation}
where $df_t=n-1$ and $df_e=n-p-1$ 
\end{itemize}

}

\frame{
\frametitle{Analysis of Variances (ANOVA)} 

What is a nested model?
\begin{itemize}
\item Model $M_1$ is a nested model of model $M_2$ when model $M_2$ contains $M_1$.
\item $M_1$ is a subset of $M_2$: $M_1 \subset M_2$
\item Example: $M_1 = bmi$ and $M_2 = bmi + map$
\end{itemize}

\centering \includegraphics [scale=0.36]{E:/Postdoc Imperial/Lectures/2023_AdvancedRegression/AdvancedRegression2022-2023/lectures/graphs/subset}


}





\frame{
\frametitle{Analysis of Variances (ANOVA)} 


\begin{itemize}
\item $F-$test, to compare two nested models: a `full' and a `reduced' model.
\item $M_2$ `full' model included $p_2$ predictors. 
\begin{table}
\begin{tabular}{ c | c | c }   
  & df & SS \\
\hline
Regression fit & $p_2$ & $SS_{Explained}(M_2)$ \\
Error & $n-p_2-1$ & $SS_{Error}(M_2)$ \\
\hline
Total & $n-1$ & $SS_{Total}(M_2)$ \\
\end{tabular}
\end{table} 
\item $M_1$ `reduced' model included $p_1$ predictors, where $p_1<p_2$. 
\begin{table}
\begin{tabular}{ c | c | c |}   
  & df & SS \\
\hline
Regression fit & $p_1$ & $SS_{Explained}(M_1)$ \\
Error & $n-p_1-1$ & $SS_{Error}(M_1)$ \\
\hline
Total & $n-1$ & $SS_{Total}(M_1)$ \\
\end{tabular}
\end{table} 
\end{itemize}


}


\frame{
\frametitle{Analysis of Variances (ANOVA)} 


\begin{itemize}
\item $F$-test, to compare two nested models:
\begin{itemize}
\item[$\diamond$]   `full' model ($M_2$) with $p_2$ parameter 
\item [$\diamond$]  `reduced' model ($M_1$) with $p_1$ parameter
\end{itemize}
\item It will always hold that:
\begin{itemize}
\item[$\diamond$] $R^2(M1) \leq R^2(M2)$
\item[$\diamond$] $SS_{Error}(M_1)\geq SS_{Error}(M_2)$ 
\item [$\diamond$] $p_2>p_1$
\end{itemize}
\item But is the `full' model ($M_2$) significantly better than a `reduced' model ($M_1$)?
\end{itemize}


}


\frame{
\frametitle{Analysis of Variances (ANOVA)} 


\begin{itemize}
\item $H_0$: Model $M_2$ fits the data as good as model $M_1$.
\begin{equation}
F = \frac{(SS_{Error}(M_1)- SS_{Error}(M_2))/(p_2-p_1)}{SS_{Error}(M_2) / (n-p_2-1)} \nonumber
\end{equation}
\item Under the Null, the test statistic $F$ follows an $F$-distribution with $(p_2-p_1)$ and $(n-p_2-1)$ degrees of freedom.
\item Interpretation 1: If we reject $H_0$, $M_2$ fits the data significantly better than model $M_1$.
\item Interpretation 2: By adding more predictors in the complex model compared to the reduced model we can explain more of the variation in $Y$.
\item \texttt{anova(M1,M2)} command in \texttt{R}.
\end{itemize}


}




\frame{
\frametitle{ANOVA example: Diabetes data} 

\centering \includegraphics [scale=0.5]{E:/Postdoc Imperial/Lectures/2023_AdvancedRegression/AdvancedRegression2022-2023/lectures/graphs/anova_diabetes1}

\begin{itemize}
\item It improves the model fit to add the variable glu to the model.
\item The `full' model ($M_2$) is better than the `reduced' model ($M_1$).
\end{itemize}


}



\subsection{Akaike criterion and other likelihood-based measures}


\frame{
\frametitle{Likelihood ratio test} 

\begin{itemize}
\item The likelihood ratio test (LRT) contrasts two nested models ($M_1 \subset M_2$). 
\item It is defined as the difference between the log-likelihoods
\begin{equation}
LRT = - 2 (\log L(M_1)- \log L(M_2)) \nonumber
\end{equation}
\item Note that $-\log L(M_2) \leq \log -L(M_1)$, \\
which is in analogy with $R^2 (M_2)\geq R^2 (M_1)$.
\item The main aim is to test if $M_2$ provides a significantly better model fit than $M_1$.
\end{itemize}

}


\frame{
\frametitle{Likelihood ratio test in R} 

\begin{itemize}
\item \texttt{library(lmtest)}
\item Function: \texttt{lrtest(M1,M2)}
\end{itemize}


\begin{itemize}
\item Is there a better model fit when including the predictor glu to the model?

\vspace{0.4cm}	
\includegraphics [scale=0.5]{E:/Postdoc Imperial/Lectures/2023_AdvancedRegression/AdvancedRegression2022-2023/lectures/graphs/LRT1}
\vspace{0.4cm}	

\item Model $M_2$ (including glu) is marginally better.
\end{itemize}

}


\frame{
\frametitle{Likelihood ratio test in R} 
\begin{itemize}
\item Is there a better model fit when including the predictor map to the model?

\vspace{0.4cm}	
\includegraphics [scale=0.5]{E:/Postdoc Imperial/Lectures/2023_AdvancedRegression/AdvancedRegression2022-2023/lectures/graphs/LRT2}
\vspace{0.4cm}	

\item Model $M_2$ (including map) is significantly better.
\item Model $M_2$ needs one more parameter to estimate than $M_1$ (degree of freedom (df)=1).
\end{itemize}

%\centering \includegraphics [scale=0.5]{E:/Postdoc Imperial/Lectures/2023_AdvancedRegression/AdvancedRegression2022-2023/lectures/graphs/LRT3}
%\begin{itemize}
%\item Switching M2 and M1 results in negative degrees of freedom.
%\end{itemize}

}



\frame{
\frametitle{Akaike information criterion (AIC)} 

\begin{itemize}
\item Akaike information criterion (AIC) combines a measure of model fit with a measure of model complexity.
\begin{equation}
AIC = -2 \log L + 2p \nonumber
\end{equation}

\begin{itemize}
\item[$\diamond$] $L$ Maximum likelihood of the model
\item[$\diamond$] $p$ Model complexity: Number of parameters in the model
\end{itemize}

\item The best model is the one with the minimum AIC value (minimum information loss). 
\item The AIC can be used for model comparison, not to assess the quality of the model fit.
\item \texttt{AIC(M1,M2)} command in R.
\end{itemize}

}




\frame{
\frametitle{Bayesian information criterion (BIC)} 

\begin{itemize}
\item Also the Bayes information criterion (BIC) combines a measure of model fit with a measure of model complexity.
\begin{equation}
BIC = -2 \log L + \log(n) p \nonumber
\end{equation}
\begin{itemize}
\item[$\diamond$]  $L$ Maximum likelihood of the model
\item[$\diamond$]  $p$ Model complexity: Number of parameters in the model
\end{itemize}
\item The best model is the one with the minimum BIC value (minimum information loss). 
\item The BIC can be used for model comparison, not to assess the quality of the model fit.
\item \texttt{BIC(M1,M2)} command in R.
\end{itemize}


}


\frame{
\frametitle{AIC and BIC} 

\begin{itemize}
\item More generally, we can understand information criteria (IC) as a compromise between model fit and model complexity
\begin{equation}
IC = -2log L + k \times p \nonumber
\end{equation}
\begin{itemize}
\item[$\diamond$]  $L$ Model fit: Maximum likelihood of the model
\item[$\diamond$]  $p$ Model complexity: Number of parameters in the model
\end{itemize}
\item The best model is the one with the minimum IC value (minimum information loss). 
\item $k$ defines the penalty of the model complexity
\begin{itemize}
\item[$\diamond$] AIC:  $k=2$ 
\item[$\diamond$] BIC:  $k=\log(n)$ 
\end{itemize}
\end{itemize}

}

\frame{
\frametitle{AIC and BIC in R} 


\centering \includegraphics [scale=0.5]{E:/Postdoc Imperial/Lectures/2023_AdvancedRegression/AdvancedRegression2022-2023/lectures/graphs/AIC1}

\begin{itemize}
\item There is no consensus between AIC and BIC.
\item Using the AIC we would prefer $M_2$, but using the BIC we would prefer $M_1$.
\item This is not a strong evidence that adding the variable glu (glucose) has a lot of benefit.
\end{itemize}

}

\frame{
\frametitle{AIC and BIC in R} 

\centering \includegraphics [scale=0.5]{E:/Postdoc Imperial/Lectures/2023_AdvancedRegression/AdvancedRegression2022-2023/lectures/graphs/AIC2}

\begin{itemize}
\item In contrast the variable map (blood pressure) greatly improves the model fit.
\item Since the variable map (blood pressure) is supported by both methods we have greater confidence that it improves the model fit.
\end{itemize}


%\centering \includegraphics [scale=0.5]{E:/Postdoc Imperial/Lectures/2023_AdvancedRegression/AdvancedRegression2022-2023/lectures/graphs/anova_diabetes2}


}



\frame{
\frametitle{How to decide which models to test?} 

\begin{itemize}
\item Backward selection 
\begin{enumerate}
\item Start with the full model and include all $p$ variables available.
\item Identify the variable with the weakest evidence and remove it.
\item Evaluate the model with $p-1$ variables.
\item Identify the variable with the weakest evidence and remove it.
\item ...
\end{enumerate}
\item Forward selection 
\begin{enumerate}
\item Start to evaluate all models including just a single predictor variable.
\item Identify the variable with the strongest univariable impact.
\item Evaluate all models including the best single predictor variable and one additional variable.
\item Identify the tuple of two variables with the strongest impact.
\item ...
\end{enumerate}
\end{itemize}

}




\frame{
\frametitle{Alternatives for model selection} 


\begin{block}{Warning!}
Backward and forward selection do rarely agree. They are highly instable and there is no guaranty that they find the optimal model. It is not recommended to use them.
\end{block}

Alternatives for model selection:

\begin{itemize}
\item Evaluate all possible models: Becomes computationally infeasible even with a moderate number of variables. 
\begin{itemize}
\item  $p=10$ variables have $2^{10} =1,024$ possible models
\item  $p=20$ variables have $2^{20} =1,048,576$ possible models
\end{itemize}
\item Penalised regression (Lecture 3b)
\end{itemize}

}


\end{document}




